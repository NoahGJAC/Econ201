\subsection*{16.4}
In every case except for public goods, there is a market mechanism that could come up with an efficient
way to allocate limited resources from those who value it the goods the least to those who value it the most.
\par
One of the roles of the government is to make sure that the distribution of goods is equitable.
Society has to decide the amount of efficiency to sacrifice for equitable distributions.
\par
The government might exist merely because society wants it to exist. They want the government deal with the distribution of goods.
Society might not trust the private market to do a job, so they have the government do it.
\par
Another reason why the government may need to play a role in the market, consider education. All children must be in school till a certain age.
They are protecting the children from their parents who might not want to send them to school.
\par 
We may need to even be protected from ourselves.
\par
There are some goods where the market mechanism shouldn't exist. For example, the market for grades.