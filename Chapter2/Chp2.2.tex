\subsection{}

What's the process of presenting findings in \emph{Economic Analysis}?

Start with \emph{observations}.
As the world changes, our observations change with it.
We then develop \emph{theories} based on these observations.

\begin{example}
    We observe that every crisis leads to a rebound.
\end{example}

We then develop a theory into a \emph{model}. These models are mathematical.
Models are simplifications of reality.
The more realistic the model, the more accurate and the more complex it is.
Models have response, independent and dependent variables.

Within the model, there are some variables that are determined within the model itself, some are outside that we drop in and utilize.
An outside variable is called an \emph{exogenous or independent variable}. 
An inside variable is called an \emph{endogenous or dependent variable}, it is determined within the model. Given some parameters, we can this will determine a particular value of this variable in the model.
The more endogenous variables, the more complex the model.

In this course, you should be able to differentiate between exogenous, endogenous, indepedent and dependent variables.

Models are based on assumptions.
Recall that the Robinson Crusoe model was based on assumptions.

Why do economists disagree?\\
They disagree because they are making different assumptions which lead to different conclusions. To each party their assumptions are correct.
The role then is to make value judgements, ask: was this a positive or normative situation?

When drawing conclusions, be careful of what you are identifying:
\begin{itemize}
    \item \begin{definition} \emph{correlation} is a relationship between two variables but the relationship is not clear.\end{definition}
    \item \begin{definition} \emph{causation} is a relationship between two variables where one causes the other.\end{definition}
\end{itemize}
\begin{example}
    Women's skirts were worn higher when stock markets were up. This is a correlation. Both the stock market and skirt height were related to economic confidence. The stock market did not cause the skirt height to rise or vice versa.
\end{example}
\begin{example}
    Raising bank interest rates reduce consumer and business spending. This is causation. The bank interest rates caused the spending to decrease.
\end{example}
The causation or correlation relationship of some variables today may change tomorrow.