\subsection{}

There are two ways to express economic statements:
\begin{itemize}
    \item \begin{definition}
        \emph{Positive statements} are factual statements. They do not always have to be factually correct.
        They just have to be presented as facts.
    \end{definition}
    \item \begin{definition}
        \emph{Normative statements} are value judgements or opinionated.
    \end{definition}
\end{itemize}
We do not need to worry about muddled statements that could be both positive and normative.
Neither positive nor normative statements are better than the other.
\begin{example}
    \begin{itemize}
        \item Positive: Today is Monday.\\
        Note: Whether or not today is Monday is not the point. The point is that it is presented as a factual statement.
        \item Normative: The minimum wage in Quebec is too low.
    \end{itemize}
\end{example}